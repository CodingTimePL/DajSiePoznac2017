\section{Opis ogólny}
\subsection{Nazwa programu}
Program będzie nosił nazwę: \textbf{MagazineManager}. Dobór nazwy umotywowany jest przeznaczeniem oprogramowania. 

\subsubsection{Poruszany problem}
Autor oprogramowania jest czytelnikiem czasopism: ,,Elektronika dla Wszystkich'', ,,Młody technik'' oraz ,,Delta''. Z czasem, kiedy egzemplarzy jest coraz więcej, trudno poszukiwać treści zamieszczonych w poprzednich numerach czasopisma, szczególnie warto wracać do opisanych tam projektów.

Aplikacja ma zarządzać gromadzonymi czasopismami, artykułami w nich zawartymi oraz ułatwić do nich dostęp, poprzez umożliwienie wyszukiwania tekstów, tagowania artykułów, dodawanie do ulubionych. 

\subsection{Użytkownik docelowy}
Program głównie będzie używany przez autora programu, jednak źródła będą udostępnione i każdy użytkownik -- pasjonata będzie mógł skorzystać z oprogramowania.

\section{Opis funkcjonalności}
\subsection{Korzystanie z programu}
Program będzie uruchamiany ze spakowanego pliku o rozszerzeniu \textbf{.jar}. Do poprawnego funkcjonowania programy potrzebny jest również serwer bazy mySQL, np. XAMPP lub WAMP. Jest on odpowiedzialny z obsługę bazy danych.

\subsection{Możliwości programu}
Program ma umożliwić realizację następujących funkcjonalności:
\begin{itemize}
\item CRUD dla dowolnego tytułu czasopisma -- formularz dodawania należy stworzyć tak, aby był on jak najbardziej uniwersalny, co umożliwi na zarządzanie zasobami innych czasopism niż wymienione powyżej,
\item CRUD zawartości czasopisma,
\item możliwość tagowania artykułów i ich wyszukiwania po tagach,
\item możliwość możliwość kategoryzowania artykułów i wyszukiwanie ich po kategoriach (kategorie takie jak kategorie artykułów w czasopismach) - dodawanie, poprzez odpowiedni formularz, kategorii według jakich artykuły w czasopiśmie są poukładane,
\item oznaczanie artykułów jako ulubione -- umożliwienie użytkownikowi łatwego odszukania interesujących czy potrzebnych mu treści,
\item eksportowanie bazy danych do pliku z poziomu aplikacji,
\item importowanie bazy danych z pliku o rozszerzeniu .csv,
\item możliwość wygenerowania z poziomu programu bazy danych potrzebnej do sprawnego działania aplikacji,
\item modyfikowanie parametrów połączenia z bazą danych,
\item generowanie list artykułów do pliku PDF.
\end{itemize}
CRUD -- od angielskiego create, read, update, delete.

\section{Format danych i struktura plików}
\subsection{Struktura katalogów}
Program będzie upakowany do pliku uruchamialnego o rozszerzeniu \textbf{.jar}. Zatem nie będzie wymagana żadna charakterystyczna struktura katalogów. Program będzie umożliwiał wybór lokalizacji zapisu i wyboru nazwy plików, które mogą zostać wygenerowane. 

\subsection{Przechowywanie danych}
Dane potrzebne do działania programu będą pobierane ze współpracującej z aplikacją bazy danych. Obsługę bazy danych będzie realizował system mySQL rozwijany przez firmę Oracle. Umożliwi to na korzystanie z narzędzi np. XAMPP i WAMP, które zawierają w sobie ten system. Narzędzia te są bardzo popularne i łatwe w obsłudze. Schemat tabel i relacji pomiędzy nimi zostanie utworzony w kolejnych etapach projektowania.

\subsection{Dane wejściowe i wejściowe}
W programie nie będzie możliwości wczytywania danych z plików zewnętrznych. 

Dane wyjściowe mogą być zapisywane do:
\begin{itemize}
\item plików o rozszerzeniu \textbf{.pdf},
\item plików eksportowanych z bazy danych o rozszerzeniu \textbf{.csv}.
\end{itemize}

\section{Testowanie}
Działanie programu zostanie przetestowane w następujący sposób:
\begin{itemize}
\item GUI programu będzie testowane w trakcie tworzenia bezpośrednio przez programistę,
\item logika aplikacji będzie testowana przez specjalnie przygotowane do tego testy.
\end{itemize}
Sama aplikacja zostanie napisana zgodnie z metodyką programowania zwinnego \textbf{Test--Driven Development}, tzn. że na początku zostaną napisane testy funkcjonalności, które mają być dostarczane przez program. 