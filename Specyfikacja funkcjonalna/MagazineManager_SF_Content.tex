\section{Opis ogólny}
\subsection{Nazwa programu}
Program będzie nosił nazwę: \textbf{MagazineManager}. Dobór nazwy umotywowany jest przeznaczeniem oprogramowania. 

\subsubsection{Poruszany problem}
Autor oprogramowania jest czytelnikiem czasopism: ,,Elektronika dla Wszystkich'', ,,Młody technik'' oraz ,,Delta''. Z czasem, kiedy egzemplarzy jest coraz więcej, trudno poszukiwać treści zamieszczonych w poprzednich numerach czasopisma, szczególnie warto wracać do opisanych tam projektów.

Aplikacja ma zarządzać gromadzonymi czasopismami, artykułami w nich zawartymi oraz ułatwić do nich dostęp, poprzez umożliwienie wyszukiwania tekstów, tagowania artykułów, dodawanie do ulubionych. 

\subsection{Użytkownik docelowy}
Program głównie będzie używany przez autora programu, jednak źródła będą udostępnione i każdy użytkownik -- pasjonata będzie mógł skorzystać z oprogramowania.

\section{Opis funkcjonalności}
\subsection{Korzystanie z programu}
Program będzie uruchamiany ze spakowanego pliku o rozszerzeniu \textbf{.jar}. 

\subsection{Możliwości programu}
Program ma umożliwić realizację następujących funkcjonalności:
\begin{itemize}
\item dodawanie zawartości czasopisma do bazy danych współpracującej z aplikacją przez zaimplementowany formularz,
\item eksportowanie bazy danych do pliku z poziomu aplikacji,
\item możliwość tagowania artykułów i ich wyszukiwania po tagach
\item możliwość możliwość kategoryzowania artykułów i wyszukiwanie ich po kategoriach,
\item oznaczanie artykułów jako ulubione,
\item generowanie list artykułów do pliku PDF.
\end{itemize}

\section{Format danych i struktura plików}
\subsection{Struktura katalogów}
Program będzie upakowany do pliku uruchamialnego o rozszerzeniu \textbf{.jar}. Zatem nie będzie wymagana żadna charakterystyczna struktura katalogów. Program będzie umożliwiał wybór lokalizacji zapisu i wyboru nazwy plików, które mogą zostać wygenerowane. 

\subsection{Przechowywanie danych}
Dane potrzebne do działania programu będą pobierane ze współpracującej z aplikacją bazy danych. \textbf{TODO: Przemyśleć rozwiązanie bazy danych i uzupełnić.} 

\subsection{Dane wejściowe i wejściowe}
W programie nie będzie możliwości wczytywania danych z plików zewnętrznych. 

Dane wyjściowe mogą być zapisywane do:
\begin{itemize}
\item plików o rozszerzeniu \textbf{.pdf},
\item plików eksportowanych z bazy danych o rozszerzeniu \textbf{.csv}.
\end{itemize}

\section{Testowanie}
Działanie programu zostanie przetestowane w następujący sposób:
\begin{itemize}
\item GUI programu będzie testowane w trakcie tworzenia bezpośrednio przez programistę,
\item logika aplikacji będzie testowana przez specjalnie przygotowane do tego testy.
\end{itemize}
Sama aplikacja zostanie napisana zgodnie z metodyką programowania zwinnego \textbf{Test--Driven Development}, tzn. że na początku zostaną napisane testy funkcjonalności, które mają być dostarczane przez program. 